%-------------------------------------------------------------------------------
%-------------------------------------------------------------------------------
%-------------------------------------------------------------------------------
% Файл типового отчёт в latex содержит титульный лист и автоматический ввод исходников в отчёт


%-------------------------------------------------------------------------------
%-------------------------------------------------------------------------------
%-------------------------------------------------------------------------------
\documentclass[12pt, a4paper]{article}
\usepackage[russian]{babel}
\usepackage{fontspec}
\setsansfont{Calibri}
\setmonofont{Consolas}
\setmainfont[
    Ligatures=TeX,
    Extension=.otf,
    BoldFont=cmunbx,
    ItalicFont=cmunti,
    BoldItalicFont=cmunbi,
]{cmunrm}
\usepackage{polyglossia}
\setdefaultlanguage{russian}
\setotherlanguage{english}


\usepackage{geometry}
\usepackage{pgfplotstable}
\usepackage[utf8]{inputenc}
 
\setlength{\parindent}{0.7em}
\setlength{\parskip}{0.7em}
\geometry{
margin=2cm
}


\usepackage{indentfirst}
\frenchspacing
\usepackage{booktabs}
\usepackage{arydshln}
\usepackage{amssymb}
\usepackage[fleqn]{amsmath}
\usepackage{subfigure}
\usepackage{xfrac}
\usepackage{esint}
\usepackage{mathbbol}
\usepackage[T1]{fontenc}
\usepackage{mathtools}
\usepackage{color}
\usepackage{ulem}
\usepackage{tabu}
\usepackage{multirow}
\usepackage{rotating}
\usepackage{enumitem}
\usepackage[outline]{contour}
\contourlength{1.2pt}

\usepackage{tikz}
\usepackage{graphics}
\usepackage{xcolor}

\usepackage{pgfplots}
\usepackage{pgfplotstable}
\usepackage{float}
\usepackage[at]{easylist}

\DeclareMathOperator{\sign}{sign}





%-------------------------------------------------------------------------------
%-------------------------------------------------------------------------------
%-------------------------------------------------------------------------------
% Титульный лист
% пример: \insertTitle{Предмет}{Тема}{Группа}{ФИО}{Вариант}{Год}
\newcommand{\insertTitle}[6]{
\begin{titlepage}
    \begin{center}
        \large
        Министерство науки и высшего образования Российской Федерации
        
        Новосибирский государственный технический университет
        
        Кафедра теоретической и прикладной информатики
        %\vspace{0.25cm}
        \vfill
        \textbf #1
        
        Курсовая работа по теме:
        
        #2
        \vfill
    \end{center}
    
    \begin{tabular}{ m{7em}  m{7em} }
        Факультет:  & ФПМИ \\ 
        Группа:     & #3 \\  
        Студент:    & #4 \\
        Вариант:    & #5
    \end{tabular}
    \vfill

    \begin{center}
        Новосибирск

        #6
    \end{center}
\end{titlepage}
}



\usepackage[utf8]{inputenc}
\usepackage{listings}
\usepackage{color}
 
\definecolor{codegreen}{rgb} {0.00, 0.60, 0.00}
\definecolor{codegray}{rgb}  {0.50, 0.50, 0.50}
\definecolor{codepurple}{rgb}{0.58, 0.00, 0.82}
\definecolor{backcolour}{rgb}{0.98, 0.98, 0.98}
 
\lstdefinestyle{mystyle}{
    backgroundcolor=\color{backcolour},   
    commentstyle=\color{codegreen},
    keywordstyle=\color{blue},
    basicstyle=\fontsize{10}{12}\selectfont\ttfamily,
    numberstyle=\tiny\color{codegray},
    stringstyle=\color{codepurple},
    breakatwhitespace=false,         
    breaklines=true,                 
    captionpos=b,                    
    keepspaces=true,                 
    numbers=left,                    
    numbersep=5pt,                  
    showspaces=false,                
    showstringspaces=false,
    showtabs=false,                  
    tabsize=4
}
\lstset{ style=mystyle}


%-------------------------------------------------------------------------------
%-------------------------------------------------------------------------------
%-------------------------------------------------------------------------------
% Считывание файла исходного кода
% пример: \myCodeInput{c++}{../main.cpp}{main.cpp}
\newcommand{\myCodeInput}[3]
{
    {\bf #2}
    \lstinputlisting[language=#1]{#3}
}


\setlength{\abovedisplayskip}{0.5pt}
\setlength{\belowdisplayskip}{0.5pt}



%-------------------------------------------------------------------------------
%-------------------------------------------------------------------------------
%-------------------------------------------------------------------------------

\begin{document}

%-------------------------------------------------------------------------------
%-------------------------------------------------------------------------------
%-------------------------------------------------------------------------------
% Дополнительные комманды для данной работы:

\newcommand{\rb}[1]{ \left(#1\right) }


%-------------------------------------------------------------------------------
%-------------------------------------------------------------------------------
%-------------------------------------------------------------------------------
\insertTitle{Планирование и анализ эксперимента}{4}{ПМ-63}{Кожекин М.В.}{Майер В.А.}{Назарова Т.А.}{Утюганов Д.С.}{9(1)}{2020}


%-------------------------------------------------------------------------------
\section{Цель работы}
Изучить методы оптимального планирования эксперимента при нелинейной параметризации функции отклика.


%-------------------------------------------------------------------------------
\section{Задание}

1. Изучить понятия локально-оптимального планирования и информационной
матрицы при нелинейной параметризации функции отклика, ознакомиться с видом
производственной функции Кобба-Дугласа.

2. По заданному типу технологии сформировать имитационную модель в виде
производственной функции Кобба-Дугласа. При этом задать истинные значения
для параметров, нелинейно входящих в модель. Выход модели зашумить, уровень
шума установить в пределах 15...20 \% от мощности полезного сигнала.

3. Выбрать план для затравочног оэксперимента, состоящий из небольшого числа
наблюдений, и смоделировать на его основе экспериментальные данные.

4. Оценить параметры модели по полученным экспериментальным данным. Для
этого необходимо перейти к линейной модели, воспользовавшись логарифмическим
представлением уравнения модели наблюдения. Параметры преобразованной модели
тогда можно оценить обычным «линейным» МНК.

5. Построить локально-оптимальный план эксперимента для исходной нелинейной
модели, воспользовавшись разработанной ранее программой синтеза дискретных
оптимальных планов и полученными оценками параметров модели. Число наблюдений
должно в 4...5 раз превышать число параметров модели.

6. По сформированной ранее (п. 2) имитационной модели провести имитационный
эксперимент в точках полученного локально-оптимального плана. Провести оценку
параметров и вычислить норму отклонения оценок от их истинны хзначений.
Вычислительный эксперимент повторить не менее 100  раз, каждый раз с новой
реализацией помехи. Вычислить среднее значение нормы отклонения оценок.
Процедуру повторить, используя в качестве плана эксперимента случайно расположенные
точки в факторном пространстве. В серии вычислительных экспериментов случайный
план фиксируется (выбирается один раз). Сделайте вывод об эффективности оптимального
планирования эксперимента для идентификации заданной нелинейной модели.

7. Оформить отчет, включающий в себя постановку задачи, оценки параметров
по затравочному эксперименту, полученный локально-оптимальный план, результаты
проведенного в п. 6  исследования и текст программы.

8. Защитить лабораторную работу. 
\vspace{20mm}



%-------------------------------------------------------------------------------
\section{Анализ}

Технология Кобба-Дугласа. Ресурсов два, изменяются в пределах [1, 10].
Постоянная отдача от масштаба. Локально-D-оптимальное планирование.
\vspace{5mm}

Модель наблюдения за объектом представляет собой уравнение вида:
\[y = \eta \rb{x, \Theta} + e, \quad \text{ где: } \]

y - значение отклика;

$\eta$ - нелинейная функция  вектора параметров $\Theta$;

$\Theta$ - вектор неизвестных параметров;

e - ошибка наблюдения
\vspace{5mm}


Функция Кобба-Дугласа $\eta \rb{x, \Theta}$ имеет вид:
\[y = \Theta_0 \cdot X_1^{\Theta_1} \cdot X_2^{\Theta_2}\]

Её логарифмическое представление:
\[y = \log(\Theta_0) + \Theta_1 \cdot \log(X_1) + \Theta_2 \cdot \log(X_2)\]

При постоянной отдаче от масштаба параметры удовлетворяют ограничению
\[\sum_{i=1}^k \Theta_i = 1\]
\vspace{5mm}


\textbf{Информационная матрица Фишера} для нелинейной модели зависит от $\hat{\Theta}$.
Приближенное значение нормированной информационной матрицы дискретного плана можно
вычислить по формуле
\[M \rb{\varepsilon_N, \hat{\Theta}} \approx M \rb{\varepsilon_N, \Theta_{true}} = 
\sum_{j=1}^n{p_j f\rb{x_j, \Theta_{true}} f^T\rb{x_j, \Theta_{true}}}, \quad \text{где}\]
\[f \rb{x, \hat{\Theta}} = \left.\frac{\partial \eta \rb{x, \Theta}}{\partial \Theta} \right|_{\Theta = \hat{\Theta}} = 
\rb{X_1^{\Theta_1} \cdot X_2^{\Theta_2},
\Theta_0 \cdot \Theta_1 \cdot X_1^{\Theta_1 - 1} \cdot X_2^{\Theta_2}, 
\Theta_0 \cdot X_1^{\Theta_1} \cdot \Theta_2 \cdot X_2^{\Theta_2 - 1}}^T\]

\textbf{Дисперсионная матрица} определяется как обратная к информационной, т.е.
\[D \rb{\varepsilon_N, \hat{\Theta}} = M^{-1} \rb{\varepsilon_N, \hat{\Theta}}\]
\pagebreak


Пусть $\Theta_{true}$ = [0.2, 0.4, 0.4], шум $\rho$ = 20\%, N = 3x5 = 15 точек, $\varepsilon$ = 1e-4.
\vspace{10mm}



\begin{tabular}{ll}
	% \toprule
	\textbf{Оптимальный план} & \textbf{Случайный план} \\
	\midrule
	\includegraphics[width=0.48\textwidth]{../pics/opt_plan.png} &  \includegraphics[width=0.48\textwidth]{../pics/rand_plan.png} \\
	RSS:  50.6925 &  RSS:  50.7152 \\
	отклонение нормы оценок:  0.0026 &  отклонение нормы оценок:  0.0027 \\
	% \bottomrule
\end{tabular}
\vspace{10mm}



\textbf{Вывод:}

Построение оптимального дискретного плана повышает точность оценки параметров
и, соответственно, точность модели. 

%-------------------------------------------------------------------------------
\section{Исходный код программы}
\myCodeInput{python}{PlanOfExp4.py}{../PlanOfExp4.py}



\end{document}