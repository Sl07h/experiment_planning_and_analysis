%-------------------------------------------------------------------------------
%-------------------------------------------------------------------------------
%-------------------------------------------------------------------------------
% Дополнительные комманды для данной работы:

\newcommand{\insertImage}[1]
{
	\begin{figure}[!htb] %htbp!
		\centering
		\includegraphics[width=0.7\textwidth]{#1}
	\end{figure}
}

\newcommand{\insertTwoImages}[2]
{
	\begin{figure}[!htb] %htbp!
		\centering
		\includegraphics[width=0.49\textwidth]{#1}
		\includegraphics[width=0.49\textwidth]{#2}
	\end{figure}
}

\newcommand{\insertFourImages}[5]
{
	\begin{figure}[!htb] %htbp!
		\centering
		\includegraphics[width=0.48\textwidth]{../pics/plan_Fedorov_30_#1_0.01_#2.png}
		\includegraphics[width=0.48\textwidth]{../pics/plan_Fedorov_30_#1_0.01_#3.png}
		\includegraphics[width=0.48\textwidth]{../pics/plan_Fedorov_30_#1_0.01_#4.png}
		\includegraphics[width=0.48\textwidth]{../pics/plan_Fedorov_30_#1_0.01_#5.png}
	\end{figure}
}


%-------------------------------------------------------------------------------
%-------------------------------------------------------------------------------
%-------------------------------------------------------------------------------
\insertTitle{Планирование и анализ эксперимента}{3}{ПМ-63}{Кожекин М.В.}{Майер В.А.}{Назарова Т.А.}{Утюганов Д.С.}{9(1)}{2020}


%-------------------------------------------------------------------------------
\section{Цель работы}
Изучить алгоритмы, используемые при построении дискретных оптимальных планов эксперимента.


%-------------------------------------------------------------------------------
\section{Задание}

1.	Изучить алгоритмы построения дискретных оптимальных планов. 

2.	Разработать программу построения дискретных оптимальных планов эксперимента, реализующую заданный алгоритм.  

3.	Для числа наблюдений 20, 25, 30, 35, 40 построить оптимальные планы на каждой из сеток, указанных в варианте задания. Выбрать лучшие дискретные планы для заданного числа наблюдений.  

4.	Оформить отчет, включающий в себя постановку задачи, результаты проведенных в п. 3 исследований, текст программы.  

5.	Защитить лабораторную работу.  



%-------------------------------------------------------------------------------
\section{Анализ}

Задана двухфакторная модель на квадрате со сторонами [-1, 1].

Дискретное множество $\tilde{X}$ - сетки 10x10 и 20x20. Строить D-оптимальные планы.
Алгоритм Фёдорова. Повторные наблюдения допускаются.
\[ y = \Theta_0 + \Theta_1 \cdot x_1 + \Theta_2 \cdot x_2 + \Theta_3 \cdot x_1 \cdot x_2 + \Theta_4 \cdot x_1^2 + \Theta_5 \cdot x_2^2 \]


Этапы {\bf алгоритма Фёдорова} синтеза непрерывного оптимального плана:

1. Выбирается невырожденный начальный план $\varepsilon^0_N$ и малая константа $\delta>0$, s = 0.

2. Выбирается пара точек: $x_j^s$, принадлежащая плану $\varepsilon^s_N$, и $x^s$, не принадлежащая плану, по правилу
\[ \left(x_j^s, x^s\right) = arg \left( \max_{x_j \in \varepsilon^s_N} \max_{x \in \tilde{X}} \Delta(x_j, x) \right) \]

где
\[ \Delta(x_j, x) = \frac{1}{N} \left[ d(x, \varepsilon_N) - d(x_j, \varepsilon_N) \right] \]
\[ - \frac{1}{N^2} \left[ d(x, \varepsilon_N) d(x_j, \varepsilon_N) - d^2 (x, x_j, \varepsilon_N) \right] \],
\[ d(x, \varepsilon) = f^T(x) M^{-1} f(x), \quad d(x,x_j, \varepsilon) = f^T(x) M^{-1} f(x_j) \]

3. Величина $\Delta(x_j, x)$ сравнивается с $\delta$. Если $\Delta(x_j, x) \le \delta$,
то вычисления прекращаются, в противном случае осуществляется переход на шаг 4.

4. Точка $x_j$ заменяется в плане на точку x. В результате получается новый план 
$\varepsilon_N^{s+1}$. Далее s  заменяется s+1 и осуществляется переход на шаг 2.

Оптимизационная процедура, выполнимая на шаге 2, может оказаться слишком трудоёмкой 
в ввычислительном плане, поэтому на практике ограничиваются поиском первой пары точек
$ \left( x_j^s, x^s \right) $, для которой выполняется условие 
$ \Delta \left( x_j^s, x^s \right) \geq \delta $. После чего выполняется шаг 4.


%-------------------------------------------------------------------------------
\section{Исследования работы алгоритма}


%---------------------------------------
\subsection{Сходимость алгоритма Фёдорова при N = 30, $\delta = 0.01$ на сетке 11x11, 21x21}
\insertFourImages{10}{0}{10}{20}{40}
\vspace{30mm}

\insertImage{../pics/convergence_Fedorov_30_11_0.01.png}

\subsection{Сходимость алгоритма Фёдорова при N = 30, $\delta = 0.01$ на сетке 21x21}

\insertFourImages{20}{0}{10}{20}{40}


Исходя из графика можно понять, что критерий D-оптимальности уменьшается, значит, наш алгоритм всё ещё сходится.
\vspace{30mm}

\insertImage{../pics/convergence_Fedorov_30_21_0.01.png}

%---------------------------------------
\subsection{Влияние шага метода}

Как видно из графиков при уменьшении $\delta$ метод сходится медленнее.
При увеличении данного параметра метод может разойтись.
\insertTwoImages{../pics/research_delta_21x21_20.png}{../pics/research_delta_21x21_40.png}


%---------------------------------------
\subsection{Влияние числа узлов плана}

При малом числе точек плана метод становится неустойчивым.
Особенно это заметно если множество $\tilde{X}$ также имеет малое число точек
\vspace{30mm}

\insertTwoImages{../pics/research_N_11x11.png}{../pics/research_N_21x21.png}


%---------------------------------------
\subsection{Влияние числа узлов сетки}

При увеличиении числа точек сетки метод становится стабильнее, но медленнее сходится.
При малых N и n метод расходится.
\insertImage{../pics/research_width.png}



%-------------------------------------------------------------------------------
\section{Исходный код программы}
\myCodeInput{python}{lab3.py}{../lab3.py}